\documentclass[12pt, a4paper, oneside]{ctexart}
\usepackage{amsmath, amsthm, amssymb, appendix, bm, graphicx, hyperref, mathrsfs, graphicx}
\usepackage{authblk}
% \usepackage[perpage]{footmisc}
\usepackage{multirow}
\usepackage{hyperref}
\hypersetup{hidelinks,
	colorlinks=true,
	allcolors=black,
	pdfstartview=Fit,
	breaklinks=true}
\usepackage{float}
\usepackage{enumerate}

% \title{\textbf{威斯敏斯特信心告白} \\ {\small \href{https://westminsterstandards.org/westminster-confession-of-faith/}{The Westminster Confession of Faith}}}
% \author[1]{译者:Job Zhao}
% \date{\today}
\linespread{1.5}

\renewcommand{\abstractname}{\Large\textbf{绪言}}
\newcounter{parnum}[section]
\newcommand{\N}{%
   \noindent\refstepcounter{parnum}%
    \makebox[1em][l]{\small{\arabic{parnum}}}}
% Use a generous paragraph indent so numbers can be fit inside the
% indentation space.
\setlength{\parindent}{2em}

\begin{document}

\setcounter{page}{0}
\begin{center}
	\Large{\textbf{创世纪}} \\ \vspace{2em}
	\small{\href{https://biblehub.com/bsb/genesis/}{Genesis}} \\ \vspace{1em}
	\small{译者:Job Zhao} \\ \vspace{1em}
	\small{编译日期: \today} \\ \vspace{1em}
	\small{Version 2.00} \\ \vspace{1em}
\end{center}
\vfill
\vspace{28em}
\begin{tabular*}{\textwidth}{cc}
	\includegraphics{figure/by-nc.eps}
	& \begin{minipage}[b]{0.6\textwidth}
		\footnotesize
		本作品采用知识共享署名-非商业性使用4.0国际许可协议进行许可。访问\url{http://creativecommons.org/licenses/by-nc/4.0/ }查看该许可协议。
	\end{minipage}
\end{tabular*}
\vspace{2mm}
\thispagestyle{empty}

\newpage

\begin{abstract}
    
\end{abstract}

\newpage
\pagenumbering{Roman}
\setcounter{page}{1}
\tableofcontents
\newpage
\setcounter{page}{1}
\pagenumbering{arabic}

\section*{介绍}


\newpage

% Genesis 1
\section*{第一章}

% The Creation

\subsection*{创世}
% In the beginning God created the heavens and the earth.
\N 起初上帝创造诸天和地。
% Now the earth was formless and void, and darkness was over the surface of the deep. And the Spirit of God was hovering over the surface of the waters.
\N 现在地是混沌空虚的,黑暗在深渊的表面之上,而上帝的圣灵在诸水上盘旋着。

\subsubsection*{第一日}
% And God said, “Let there be light,” and there was light.
\N 而上帝说,“让那有光”,就有了光。
% God saw that the light was good, and he separated the light from the darkness.
\N 上帝见那光是善的,祂就将那光从那黑暗分离。
% God called the light “day,” and the darkness He called “night.” And there was evening, and there was morning—the first day.
\N 上帝称那光为“昼”,而那黑暗祂称为“夜。”于是有夜晚,又有早晨---这是第一日。

\subsubsection*{第二日}
% And God said, “Let there be an expanse between the waters, to separate the waters from the waters.”
\N 而上帝说,“让那诸水之间有穹苍将水和水分离。”
% So God made the expanse and separated the waters beneath it from the waters above. And it was so.
\N 所以上帝制造那穹苍并将它之下的诸水与之上的诸水分离。而它就如此。
% God called the expanse “sky.” And there was evening, and there was morning—the second day.
\N 上帝称穹苍为“天空。”于是有夜晚,又有早晨---这是第二日。

\subsubsection*{第三日}
% And God said, “Let the waters under the sky be gathered into one place, so that the dry land may appear.” And it was so.
\N 而上帝说,“让天空之下的诸水被聚集到一个地方,以使陆地能显现。”而它就如此。
% God called the dry land “earth,” and the gathering of waters He called “seas.” And God saw that it was good.
\N 上帝称那陆地为“大地”,而诸水聚集处祂称为“海洋。”而上帝见它是善的。
% Then God said, “Let the earth bring forth vegetation: seed-bearing plants and fruit trees, each bearing fruit with seed according to its kind.” And it was so.
\N 然后上帝说,“让大地生出植被:结种子的植物和结带种子果实的果树,各按其类。”而它就如此。
% The earth produced vegetation: seed-bearing plants according to their kinds and trees bearing fruit with seed according to their kinds. And God saw that it was good.
\N 大地产出植被:结种子的植物按它们的种类,而结带种子的果实的树按它们的种类。而上帝见它是善的。
% And there was evening, and there was morning—the third day.
\N 于是有夜晚,又有早晨---这是第三日。
	
\thispagestyle{empty}
\newpage

\section*{后记}


\end{document}